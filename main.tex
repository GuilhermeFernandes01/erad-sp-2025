%%%%%%%%%%%%%%%%%%%%%%%%%%%%%%%%%%%%%%%%%%%%%%%%%%%%%%%%%%%%%%%%%%%%%%
% How to use writeLaTeX: 
%
% You edit the source code here on the left, and the preview on the
% right shows you the result within a few seconds.
%
% Bookmark this page and share the URL with your co-authors. They can
% edit at the same time!
%
% You can upload figures, bibliographies, custom classes and
% styles using the files menu.
%
%%%%%%%%%%%%%%%%%%%%%%%%%%%%%%%%%%%%%%%%%%%%%%%%%%%%%%%%%%%%%%%%%%%%%%

\documentclass[12pt]{article}

\usepackage{sbc-template}

\usepackage{graphicx,xurl}

\usepackage{chemformula} % para escrever CO2 corretamente

\usepackage[brazil]{babel}   
\usepackage[utf8]{inputenc}  

\usepackage[hidelinks]{hyperref}
     
\sloppy

\title{Migração de máquinas virtuais como estratégia para \textit{data centers} sustentáveis}

\author{Guilherme Fernandes Moraes da Silva\inst{1}, Daniel Cordeiro\inst{1}}

\address{Escola de Artes, Ciências e Humanidades -- Universidade de São Paulo (USP)\\ -- São Paulo, SP -- Brasil
  \email{\{guilherme.fernandes01,daniel.cordeiro\}@usp.br}
}

\begin{document} 

\maketitle

\begin{abstract}
The growing use of geo-distributed data centers enhances cloud computing capacity but also raises energy consumption and greenhouse gas emissions. In this work, we investigate, through simulation, scheduling algorithms applied to these data centers to reduce environmental impact. Our proposed approach explores virtual machine (VM) migration policies to identify strategies that combine performance and sustainability. We expect the experiments to support the implementation of scheduling solutions capable of minimizing greenhouse gas emissions and fostering more sustainable operations in large-scale infrastructures.
\end{abstract}
     
\begin{resumo}
O crescente uso de \textit{data centers} geodistribuídos impulsiona a capacidade da computação em nuvem, mas também eleva o consumo de energia e as emissões de gases do efeito estufa. Neste trabalho, investigamos, por meio de simulação, algoritmos de escalonamento aplicados a esses \textit{data centers}, visando à redução do impacto ambiental. A abordagem proposta explora políticas de migração de máquinas virtuais (VMs) para indicar estratégias que aliem desempenho e sustentabilidade. A expectativa é que os experimentos subsidiem a implementação de soluções de escalonamento capazes de minimizar emissões de gases do efeito estufa e promover operações mais sustentáveis em infraestruturas de larga escala.
\end{resumo}

\section{Introdução}
\textit{Data centers} já consomem aproximadamente 2\% da eletricidade global e, junto às redes de transmissão de dados, respondem por cerca de 1\% das emissões de gases do efeito estufa associadas à energia, fenômeno potencializado pelos expressivos aumentos no armazenamento e na carga de trabalho desses centros \cite{masanet:20}. Segundo a \textit{International Energy Agency} (IEA), o crescimento é particularmente acentuado na China e nos Estados Unidos, que em 2024 somaram mais da metade da demanda mundial. Projeções\footnote{\url{https://www.iea.org/reports/electricity-2025}} indicam que o consumo chinês pode dobrar até 2027, enquanto nos EUA já atingiu 176 TWh em 2023 (4,4\% do total do país) e pode chegar a 325–580 TWh em 2028 (6,7\% a 12\% do total do país) \cite{shehabi:24}.

Uma estratégia promissora para mitigar as emissões de gases do efeito estufa nos \textit{data centers} consiste no desenvolvimento de algoritmos de escalonamento cientes de carbono (\textit{carbon aware}), que minimizam essas emissões~\cite{vasconcelos2023optimal}. Quando aplicada em \textit{data centers} geodistribuídos, com acesso à produção de energia renovável e a diferentes matrizes energéticas, essa abordagem potencializa a   . A avaliação dessas estratégias em ambiente real é onerosa; por isso, simuladores oferecem uma plataforma controlada para modelar diferentes cenários, comparar políticas e orientar melhorias de desempenho e sustentabilidade.

Diante desse contexto, este trabalho propõe a realização de experimentos com simuladores para identificar e avaliar algoritmos, ou combinações destes, que possam superar as soluções atualmente disponíveis, com o objetivo de reduzir o impacto ambiental por meio da minimização das emissões de gases do efeito estufa.

\section{Conceitos fundamentais}
\subsection{Computação verde}
A computação verde é um paradigma que promove soluções de tecnológicas ambientalmente sustentáveis e energeticamente eficientes. Envolve planejar e desenvolver infraestruturas e estratégias de gestão ambiental, como otimização do consumo de energia, reuso de recursos e gerenciamento eficiente, para criar produtos e serviços que demandem menos energia e emitam menos gases de efeito estufa. \cite{paul:2018}.

\subsection{Virtualização e simulação de sistemas distribuídos}
Uma máquina virtual (VM) é uma cópia isolada e eficiente de um computador físico, criada e gerida pelo hipervisor (VMM), que aloca recursos físicos do \textit{host} (máquina real) a ambientes convidados independentes, possibilitando executar sistemas operacionais e aplicações de forma segura, flexível e escalável \cite{popek:74,buyya:13}. Sobre essa base, a migração ao vivo de VMs permite transferir o estado completo (sistema operacional, aplicações e memória) de uma VM para outro \textit{host} sem interromper os serviços, simplificando manutenção de \textit{hardware}, balanceamento de carga e gerenciamento de falhas, de maneira transparente para os usuários, sem reinicializações ou redirecionamentos de aplicação \cite{clark:2005}. Para avaliar o impacto dessas técnicas em grande escala, simuladores reproduzem, em ambiente virtual controlado, o comportamento de plataformas de grande escala, mitigando os custos e riscos de testes em infraestruturas reais. Além de gerar resultados confiáveis, validados cientificamente, permitem explorar cenários pouco acessíveis na prática, contribuindo para a redução de gastos em tempo, dinheiro e energia.

\section{Desafios em simulações cientes de emissões}
A incorporação da análise de emissões de \ch{CO2} em simuladores apresenta desafios relacionados à modelagem do consumo energético. Essa tarefa requer a obtenção e integração de dados específicos que diferenciem fontes renováveis de não renováveis, além da implementação de componentes capazes de mensurar o impacto das estratégias de escalonamento na eficiência energética. Por vezes, é necessário desenvolver módulos adicionais ou realizar adaptações personalizadas nos simuladores para capturar de forma adequada essas variáveis ambientais.

De forma complementar, comparar algoritmos de escalonamento em cenários que priorizam sustentabilidade impõe desafios. Cada abordagem pode privilegiar aspectos distintos --- como latência, qualidade de serviço, custo, consumo de energia ou emissões de \ch{CO2} --- e utilizar modelos e métricas variados \cite{kumar:19}. Essa diversidade dificulta a obtenção de resultados diretamente comparáveis, principalmente quando os algoritmos são testados em simuladores diferentes. \textit{Frameworks} como o CloudSim Plus \cite{silva:17} oferecem uma base flexível que pode ser expandida com módulos de análise de emissões, facilitando a exploração de novas estratégias para otimização de recursos e redução dos impactos ambientais gerados pelos \textit{data centers}.

\section{Metodologia e resultados preliminares}

Para simular cenários de migração de VMs, foram desenvolvidos dois programas em Java utilizando o CloudSim Plus. O ambiente modela um \textit{data center} com 100 \textit{hosts}, cada qual dispondo de 4 TB de disco, 16 Gb/s de rede e \textit{cores} de 3.000 MIPS. A capacidade é heterogênea: 40 \textit{hosts} com 16 \textit{cores}/64 GB RAM; 30, 32 \textit{cores}/128 GB; 20, 64 \textit{cores}/256 GB; e 10, 128 \textit{cores}/512 GB. Instanciaram-se 150 VMs (1, 2, 4, 8, 16, 32 ou 64 \textit{cores}), todas com 16 GB RAM, 1 GB de disco e largura de banda proporcional ao total de VMs. Cada VM executa um \textit{cloudlet}, tarefa que representa uma aplicação ou parte dela, cuja carga inicia com 80\% de uso de CPU e cresce em 4\% a cada segundo.

Compararam-se duas políticas de migração: \textit{Best Fit}, que migra VMs quando um \textit{host} ultrapassa 80\% ou cai abaixo de 10\% de utilização e seleciona o \textit{host} com maior capacidade disponível que atenda aos requisitos da carga; e \textit{First Fit}, que também considera os mesmos limites de utilização, mas escolhe o primeiro \textit{host} com capacidade suficiente para a VM, desconsiderando o consumo de energia. Durante a migração, 50\% da largura de banda do \textit{host} é reservada para a transferência das VMs.

Os resultados indicam que a política \textit{Best Fit} superou a \textit{First Fit}, apresentando uma redução de cerca de 30\% no consumo energético durante a migração. Embora a vantagem no tempo total de execução tenha sido de apenas 1\% em um cenário ainda com poucos recursos, o tempo máximo de migração da \textit{Best Fit} foi aproximadamente 55\% menor em comparação à \textit{First Fit}. Esses achados evidenciam que a escolha adequada da política de migração pode gerar ganhos expressivos tanto no desempenho quanto na eficiência energética de \textit{data centers}.

O código-fonte e instruções de reprodução estão disponíveis publicamente em um repositório no GitHub\footnote{\url{https://github.com/GuilhermeFernandes01/migrations-simulations-algorithms-cloudsimplus/tree/erad-sp-2025}}. Essa disponibilização visa facilitar a reprodução dos experimentos e a ampliação dos estudos por outros pesquisadores interessados em avaliar ou aprimorar estratégias de migração de VMs em \textit{data centers}.

\section{Considerações finais}
Este trabalho mostrou que o \textit{CloudSim Plus} é adequado para comparar políticas de migração de VMs e aferir eficiência energética, embora os experimentos, limitados a 100 \textit{hosts} e 150 VMs, restrinjam a generalização dos resultados. Pesquisas futuras devem escalar o número de \textit{hosts}/VMs de acordo com traços de carga reais. O consumo energético obtido pode ser convertido em \ch{CO2eq} por meio da intensidade de carbono regional, viabilizando análises cientes de carbono sem alterar o simulador. A inclusão de métricas de \ch{CO2eq}, qualidade de serviço e custo no CloudSim Plus permitirá estudar, de forma padronizada, as relações entre desempenho, despesas e sustentabilidade em infraestruturas de larga escala. Espera-se, portanto, que o arcabouço construído e as pesquisas futuras ampliem o conhecimento e a aplicação de estratégias que minimizem emissões de gases de efeito estufa sem comprometer qualidade de serviço nem viabilidade econômica.

Este trabalho foi realizado com apoio da Fundação de Amparo à Pesquisa do Estado de São Paulo (FAPESP), procs. \mbox{\#2019/26702-8}, \mbox{\#2021/06867-2} e \mbox{\#2023/00811-0}, do CCD Cidades Carbono Neutro (FAPESP \mbox{\#2024/01115-0}), do CNPq (proc. \mbox{\#444766/2024-3}) e do grupo de pesquisa THUS (Techno-Human Systems of the Future) do Centro Internacional de Pesquisa (IRC) `Transitions' (CNRS, USP e FAPESP CIP \mbox{\#2025/01171-0}).

\renewcommand{\refname}{Referências}
\bibliographystyle{sbc}
\bibliography{sbc-template}

\end{document}
