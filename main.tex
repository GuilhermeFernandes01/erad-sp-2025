%%%%%%%%%%%%%%%%%%%%%%%%%%%%%%%%%%%%%%%%%%%%%%%%%%%%%%%%%%%%%%%%%%%%%%
% How to use writeLaTeX: 
%
% You edit the source code here on the left, and the preview on the
% right shows you the result within a few seconds.
%
% Bookmark this page and share the URL with your co-authors. They can
% edit at the same time!
%
% You can upload figures, bibliographies, custom classes and
% styles using the files menu.
%
%%%%%%%%%%%%%%%%%%%%%%%%%%%%%%%%%%%%%%%%%%%%%%%%%%%%%%%%%%%%%%%%%%%%%%

\documentclass[12pt]{article}

\usepackage{sbc-template}

\usepackage{graphicx,url}

\usepackage{chemformula} % para escrever CO2 corretamente

\usepackage[brazil]{babel}   
\usepackage[utf8]{inputenc}  

     
\sloppy

\title{Migração de máquinas virtuais como estratégia para \textit{data centers} sustentáveis}

\author{Guilherme Fernandes Moraes da Silva\inst{1}, Daniel Cordeiro\inst{1}}

\address{Escola de Artes, Ciências e Humanidades -- Universidade de São Paulo (USP)\\ -- São Paulo, SP -- Brasil
  \email{\{guilherme.fernandes01,daniel.cordeiro\}@usp.br}
}

\begin{document} 

\maketitle

\begin{abstract}
The growing use of geo-distributed data centers enhances cloud computing capacity but also raises energy consumption and greenhouse gas emissions. In this work, we investigate, through simulation, various scheduling algorithms applied to these data centers to reduce environmental impact. Our proposed approach explores virtual machine (VM) migration policies to identify strategies that combine performance and sustainability. We expect that the experiments will support the implementation of scheduling solutions capable of minimizing greenhouse gas emissions and fostering more sustainable operations in large-scale infrastructures.
\end{abstract}
     
\begin{resumo}
O crescente uso de \textit{data centers} geodistribuídos impulsiona a capacidade da computação em nuvem, mas também eleva o consumo de energia e as emissões de gases do efeito estufa. Neste trabalho, investigamos, por meio de simulação, diversos algoritmos de escalonamento aplicados a esses \textit{data centers}, visando a redução do impacto ambiental. A abordagem proposta explora políticas de migração de máquinas virtuais (VMs) para indicar estratégias que aliem desempenho e sustentabilidade. A expectativa é que os experimentos subsidiem a implementação de soluções de escalonamento capazes de minimizar emissões de gases do efeito estufa e promover operações mais sustentáveis em infraestruturas de larga escala.
\end{resumo}


\section{Introdução}
Atualmente, os \textit{data centers} representam aproximadamente 2\% do consumo energético global e, juntamente com as redes de transmissão de dados, respondem por cerca de 1\% das emissões de gases do efeito estufa associadas à energia, fenômeno potencializado pelos expressivos aumentos no armazenamento e na carga de trabalho desses centros \cite{masanet:20}. Segundo a \textit{International Energy Agency} (IEA), China e Estados Unidos registraram elevações na demanda energética impulsionadas pelos \textit{data centers}, respondendo juntos por mais da metade do consumo global em 2024. Projeções\footnote{https://www.iea.org/reports/electricity-2025} indicam que, na China, esse consumo pode dobrar até 2027, enquanto nos EUA já atingiu 176 TWh em 2023 (4,4\% do total do país) e pode chegar a 325–580 TWh em 2028 (correspondendo de 6,7\% a 12\% do consumo total do país) \cite{shehabi:24}.

Uma estratégia promissora para mitigar as emissões de gases do efeito estufa nos \textit{data centers} consiste no desenvolvimento de algoritmos de escalonamento eficientes, que minimizam essas emissões. Quando aplicada em \textit{data centers} geodistribuídos, com acesso à produção de energia renovável e a diferentes matrizes energéticas, essa abordagem potencializa a redução do impacto ambiental ao aproveitar variações regionais na disponibilidade de fontes limpas. Para tal, simuladores são amplamente utilizados, pois oferecem uma plataforma controlada que possibilita a modelagem precisa do comportamento dos sistemas. Essa abordagem permite testar diferentes cenários e configurações, avaliando o impacto de diversos algoritmos de escalonamento e facilitando a implementação de melhorias que otimizem o desempenho e a sustentabilidade das instalações.

Diante desse contexto, este trabalho propõe a realização de experimentos com simuladores para identificar e avaliar algoritmos, ou combinações destes, que possam superar as soluções atualmente disponíveis, com o objetivo de reduzir o impacto ambiental por meio da minimização das emissões de gases do efeito estufa.

\section{Conceitos fundamentais}
\subsection{Máquinas virtuais}
Uma máquina virtual (VM) é considerada uma duplicata eficiente e isolada da máquina real criado pelo hipervisor, também conhecido como monitor de máquina virtual (VMM). O VMM gerencia os recursos físicos do sistema anfitrião (a máquina real) e provisiona vários ambientes virtuais independentes, denominados convidados. Dessa forma, a VM permite a execução de sistemas operacionais e aplicações de forma independente e segura, além de evitar conflitos de recursos, como a alocação simultânea da mesma região de memória \cite{popek:74}. A virtualização relaxa as restrições de \textit{hardware}, amplia a flexibilidade do sistema e possibilita maior escalabilidade, facilitando o gerenciamento e o provisionamento de recursos em resposta às demandas dinâmicas da computação moderna \cite{buyya:13}.

\subsection{Simuladores}
Os simuladores são ferramentas fundamentais para o estudo de sistemas distribuídos, pois permitem reproduzir em ambiente virtual controlado o comportamento de plataformas de grande escala, mitigando os custos e riscos inerentes a testes em infraestruturas reais. Além de fornecer resultados confiáveis --- corroborados por extensa validação científica ---, essas ferramentas facilitam a exploração de cenários pouco acessíveis na prática, contribuindo para a redução de gastos em tempo, dinheiro e energia.

\section{Desafios em simulações cientes de emissões}

A incorporação da análise de emissões de \ch{CO2} em simuladores apresenta desafios relacionados à modelagem do consumo energético. Essa tarefa requer a obtenção e integração de dados específicos que diferenciem fontes renováveis de não renováveis, além da implementação de componentes capazes de mensurar o impacto das estratégias de escalonamento na eficiência energética. Por vezes, é necessário desenvolver módulos adicionais ou realizar adaptações personalizadas nos simuladores para capturar de forma adequada essas variáveis ambientais.

De forma complementar, comparar algoritmos de escalonamento em cenários que priorizam sustentabilidade impõe desafios. Cada abordagem pode privilegiar aspectos distintos --- como latência, qualidade de serviço, custo, consumo de energia ou emissões de \ch{CO2} --- e utilizar modelos e métricas variados \cite{kumar:19}. Essa diversidade dificulta a obtenção de resultados diretamente comparáveis, principalmente quando os algoritmos são testados em simuladores diferentes. Nesse contexto, \textit{frameworks} como o CloudSim Plus \cite{silva:17} oferecem uma base flexível que pode ser expandida com módulos de análise de emissões, facilitando a exploração de novas estratégias para otimização de recursos e redução dos impactos ambientais gerados pelos \textit{data centers}.

\section{Metodologia e resultados preliminares}

Para simular cenários de migração de VMs, foram desenvolvidos dois programas em Java utilizando o CloudSim Plus. O ambiente consiste em um \textit{data center} com 100 \textit{hosts}, cada um equipado com 4 TB de armazenamento, 16 Gb/s de largura de banda e \textit{cores} de 3.000 MIPS, com diferentes configurações de processamento e memória: 40 \textit{hosts} com 16 \textit{cores} e 64 GB de RAM, 30 com 32 \textit{cores} e 128 GB, 20 com 64 \textit{cores} e 256 GB, e 10 com 128 \textit{cores} e 512 GB. Foram criadas 150 VMs distribuídas de acordo com o número de \textit{cores} (1, 2, 4, 8, 16, 32 e 64), todas com 16 GB de RAM, 1 GB de disco e largura de banda proporcional ao total de VMs. Cada VM executa um \textit{cloudlet} --- tarefa que representa uma aplicação ou parte dela --- iniciando com 80\% de uso de CPU e aumentando esse consumo em 4\% a cada segundo.

Foram realizados dois experimentos, cada um adotando uma política diferente de migração de VMs. A primeira política baseia-se em limites estáticos de utilização de CPU dos \textit{hosts}, considerando um \textit{host} sobrecarregado quando sua utilização ultrapassa 80\% e subutilizado quando cai abaixo de 10\%. Nesta política, foi utilizado o critério \textit{Best Fit}, alocando as VMs para o \textit{host} com maior capacidade disponível que atendesse aos requisitos da carga. Durante a migração, 50\% da largura de banda do \textit{host} é reservada para a transferência das VMs, permitindo a análise do comportamento em cenários de sobrecarga e subutilização para balancear a carga e otimizar o uso dos recursos. Além dessa estratégia, também foi implementada a política \textit{First Fit}, que seleciona o primeiro \textit{host} com capacidade suficiente para a VM, desconsiderando o consumo de energia, considerando o mesmo limite de 80\% para sobrecarga e 10\% para subutilização.

Os resultados indicam que a política \textit{Best Fit} superou a \textit{First Fit}, apresentando uma redução de cerca de 30\% no consumo energético durante a migração. Embora a vantagem no tempo total de execução tenha sido de apenas 1\% em um cenário ainda com poucos recursos, o tempo máximo de migração da \textit{Best Fit} foi aproximadamente 55\% menor em comparação à \textit{First Fit}. Esses achados evidenciam que a escolha adequada da política de migração pode gerar ganhos expressivos tanto no desempenho quanto na eficiência energética de \textit{data centers}.

O código-fonte dos experimentos está disponível publicamente em um repositório no GitHub\footnote{https://github.com/GuilhermeFernandes01/migrations-simulations-algorithms-cloudsimplus}, acompanhado de instruções detalhadas sobre como configurar o ambiente de desenvolvimento, executar os testes e analisar os resultados. Essa disponibilização visa facilitar a reprodução dos experimentos e a ampliação dos estudos por outros pesquisadores interessados em avaliar ou aprimorar estratégias de migração de VMs em \textit{data centers}.

\section{Considerações finais}

Este artigo abordou a migração de VMs em \textit{data centers}, destacando como o CloudSim Plus pode servir de base para comparar diferentes políticas de migração. Verificou-se que, ao configurar a mesma infraestrutura para diversas estratégias, torna-se viável avaliar com precisão qual abordagem melhor atende a objetivos específicos, como a eficiência energética.

A implementação de métricas de emissão de gases de efeito estufa no CloudSim Plus representa um passo relevante para trabalhos futuros, pois permite avaliar como diferentes técnicas de migração e escalonamento podem (ou não) reduzir o impacto ambiental em infraestruturas de larga escala. Esses resultados devem fornecer subsídios concretos para projetistas e operadores de \textit{data centers}, incentivando práticas mais sustentáveis porque, embora o CloudSim Plus ofereça recursos valiosos para compreender a dinâmica de infraestruturas de computação distribuída, trata-se de uma ferramenta neutra em termos de sustentabilidade. Dessa forma, a redução do consumo energético depende de cenários e métricas adequadamente definidos, tornando a eficácia das estratégias de escalonamento ou migração condicionada à qualidade dos experimentos e dos parâmetros utilizados.

Ao comparar e refinar diferentes abordagens com foco em eficiência energética, torna-se possível promover práticas mais sustentáveis em \textit{data centers}. Espera-se, portanto, que o arcabouço construído e as pesquisas futuras contribuam para ampliar o conhecimento e a aplicação de técnicas que minimizem as emissões de gases de efeito estufa.

\renewcommand{\refname}{Referências}
\bibliographystyle{sbc}
\bibliography{sbc-template}

\end{document}
